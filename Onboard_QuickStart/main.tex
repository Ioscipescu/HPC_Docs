\documentclass[11pt,letterpaper]{article} 

\usepackage[T1]{fontenc}
\usepackage{amsmath}
\usepackage{amssymb}
\usepackage{amsfonts}
\usepackage{mathpazo}
\usepackage{tikz}
\usepackage{parskip}
\usepackage{enumitem}
\usepackage{sagetex}
\usepackage{fancyhdr}
% \usepackage[affil-it]{authblk}
\usepackage[onehalfspacing]{setspace}
\usepackage[letterpaper, , margin=1in]{geometry}
\usepackage{titling}
\usepackage{xcolor}
\usepackage{minted}
\usemintedstyle{perldoc}
% \usepackage[
% backend=biber,
% style=alphabetic,
% sorting=ynt
% ]{biblatex}
% \addbibresource{annot.bib}

\definecolor{LightGray}{gray}{0.9}

\usepackage{hyperref} % Needs to be the last package loaded
\hypersetup{
    colorlinks   = true,    % Colors links instead of ugly boxes
    urlcolor     = blue,    % Color for external hyperlinks
    linkcolor    = blue,    % Color of internal links
    citecolor    = red,     % Color of citations
    linktoc      = section  % Make only section number in TOC clickable
}

\setminted{
    breaklines=true,
    fontsize=\small,
    bgcolor=LightGray,
    frame=lines,
    linenos=true,
}

\newenvironment{example}{
  \subsection{Examples}
  \begin{flushleft}
}{
  \end{flushleft}
}
\newenvironment{usecases}{
  \subsection{Use Cases}
  \begin{flushleft}
}{
  \end{flushleft}
}

\def\startdoc{{
            \noindent\parbox{\linewidth}{%
                \parbox{\linewidth}{\fontsize{16}{18}\selectfont\thetitle}\hfill%
                }
            \par\noindent\rule{\textwidth}{0.4pt}
            \\
            \\}}

\date{\today}


\title{Riviera Onboard Quickstart Guide}

\begin{document}
\startdoc

\section{Connecting to Riviera}

\begin{description}
    \item[Global Protect VPN] In order to connect to the Riviera HPC system you must be connected to CSU's campus network. Remotely this can be achieved by using the Global Protect VPN documented \href{https://csusystem.freshservice.com/support/solutions/folders/23000047198}{here}.
    \item[Initial Credentials] You will be provided with initial credentials that will work to login to Riviera once. It is important to note that your username and password will be separate from the eID supplied by CSU.
    \item[Secure Shell] Once connected to the campus's network and given your inital credentials, Riviera is accessed through Secure Shell (SSH). Connecting to Riviera via SSH is achieved by running the command \mintinline{bash}|ssh your_username@riviera.colostate.edu| in your terminal of choice. Once prompted enter in the one time password supplied to you in the previous step.
    \item[Password Change] The first thing that needs to happen after logging into Riviera is a password change. To change your password, run the command \mintinline{bash}|passwd| in your SSH instance. It will prompt you to enter your old password (the one time password supplied to you) and then to enter in your new password twice. Once you have reset your password make sure to save your new password in a secure location such as your personal password manager.
\end{description}

\section{Riviera Overview}
\begin{description}
    \item[Modules] Modules are packages preloaded onto Riviera for any user to use without needing to install from source. To view a list of available modules you can run the command \mintinline{bash}|module av|. To view a list of modules you have loaded you can run the command \mintinline{bash}|module list|. In order to do any sort of computing on the cluster you will need to load the slurm module. To load the module (or any other module) run the command \mintinline{bash}|module load slurm|. To unload a module you can run the command \mintinline{bash}|module unload module_name|, and to unload all modules you can run \mintinline{bash}|module purge|.
    \item[Space] Riviera at this time does not contain any scratch space. Instead, every user's home directory is served to them by NFS. Users should not keep over 500 GB of data in their home directory long term, and start out with a limit of 1 TB of available storage in their home directory.
    \item[Available Resources] Riviera has three main types of compute nodes, CPU compute nodes, high memory CPU compute nodes, and GPU compute nodes. The CPU compute nodes each have 2 AMD EPYC 7763 64-Core Processors without hyperthreading enabled and 500 GB of available memory. The high memory compute nodes each have two AMD EPYC 7763 64-Core Processors with hyperthreading enabled and 2000 TB of available memory. The GPU compute nodes each have 2 AMD EPYC 7763 64-Core Processors without hyperthreading enabled, four Nvidia A100 80GB graphics cards, and 500 GB of available memory. Each compute node type also comes in three different run length options, they can be run with maximum 2 hour jobs, 1 day jobs, or 1 week jobs.
    \item[Data Transfer] In order to transfer data to/from Riviera either SCP or rsync should be used. Rsync should be used preferentially for transferring large amounts of data. A basic tutorial on rsync can be found \href{https://linuxize.com/post/how-to-use-rsync-for-local-and-remote-data-transfer-and-synchronization/}{here} and a basic tutorial on SCP can be found \href{https://linuxize.com/post/how-to-use-scp-command-to-securely-transfer-files/}{here}.
    \item[Slurm] Slurm is the job scheduling system run by Riviera. It can enable users to schedule tasks using SBATCH scripts or to run on a compute node interactively. To run an SBATCH script you execute the command \mintinline{bash}|sbatch my_script.sh|. To run interactively you execute the command \mintinline{bash}|srun --partition=short-cpu --pty bash|, short-cpu is the default partition to run on but any partition can be specified depending on the resources needed. Once this command is run you will be running all further commands on the node selected until exiting using the command \mintinline{bash}|exit|. 
    \item[Group Access] If a research group needs access to Riviera every user will have their own unique login instead of all users sharing a single login. From there group spaces can be set up for collaborative work. 
\end{description}

\section{Prohibited Actions}
\begin{description}
    \item[Running Code on Login] Code, including creating virtual environments, compiling, and removing large directories should be run on a compute node and not the login node. These tasks can all instead be ran interactively using the \mintinline{bash}|srun --pty bash| command which puts you in a bash environment on the short-cpu partition. Because storage is served up by NFS on Riviera any changes made in your home directory on any partition will be reflected on your home directory in all other partitions.
    \item[Connecting with an IDE] IDEs like VSCode, PyCharm, and others should not be used to connect to Riviera, the way they manage SSH leads to them taking up most of the resources available on the login node and we are not set up to handle that. If a user is not comfortable doing all of their work in the terminal using nano or vim then they can edit their files locally in their IDE of choice and transfer those files back to Riviera with SCP or rsync.
    \item[Long term file storage] Riviera is not for long term user file storage. As such, users should not permanently store files in their home directory and should instead transfer files off of Riviera back to their local machines when done with their compute. Users should also not use more than 500 GB of space on their home directory.  
\end{description}

\end{document}