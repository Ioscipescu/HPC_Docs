\begin{usecases}
    PyTorch is used for GPU Processing using python. It has built in support for cuda and can be used for general GPU compute using cuda operations or for machine learning training using libraries designed for assisting in training. It makes use of tensor objects to achieve its computation, more can be read on the PyTorch website and documentation \href{https://pytorch.org/}{here}.
\end{usecases}

\begin{example}
    \noindent\textbf{pytorch-cuda.py}\vspace{-0.5em}
    \inputminted{python3}{source_code/PyTorch/cuda/pytorch-cuda.py} 
    \pagebreak
    \noindent\textbf{pytorch-cuda.sh}\vspace{-0.5em}
    \inputminted{bash}{source_code/PyTorch/cuda/pytorch-cuda.sh}

    \noindent\textbf{pytorch-stream.py}
    \noindent\textbf{from \href{https://www.codecademy.com/resources/docs/pytorch/gpu-acceleration-with-cuda/performance-optimization}{codecademy}}\vspace{-0.5em}
    \inputminted{python3}{source_code/PyTorch/stream/pytorch-stream.py}
    \pagebreak
    \noindent\textbf{pytorch-stream.sh}\vspace{-0.5em}
    \inputminted{bash}{source_code/PyTorch/stream/pytorch-stream.sh}

    % \noindent\textbf{nlp.py}
    % \noindent\textbf{from \href{https://docs.pytorch.org/tutorials/intermediate/char_rnn_classification_tutorial.html}{PyTorch Docs}}\vspace{-0.5em}
    % \inputminted{python3}{source_code/PyTorch/nlp/nlp.py}

    % \noindent\textbf{pytorch-nlp.sh}\vspace{-0.5em}
    % \inputminted{bash}{source_code/PyTorch/nlp/pytorch-nlp.sh}
\end{example}